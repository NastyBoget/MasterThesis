\section{Введение}
\label{sec:Chapter0} \index{Chapter0}

В мире с каждым годом возрастает объём данных, при этом неструктурированные данные составляют подавляющее большинство.
По некоторым оценкам процент неструктурированных данных составляет до 80\% от общего числа~\cite{white2000enterprise}.
Современные оценки количества информации оперируют зеттабайтами (зеттабайт равен $10^{21}$ байт),
откуда становится очевидной трудность в анализе данных человеком.
Справиться с такими объёмами данных, в частности текстовых документов, может помочь их автоматический анализ.

При работе с текстовыми документами одним из первых шагов автоматической обработки документов является их логическое структурирование,
т.е. разбиение документов на некоторые смысловые части и определение взаимодействия этих частей друг с другом.
\todo[inline]{Примеры и ссылки}
В работе~\cite{dori1997representation} логическая структура документа определяется как иерархия составных частей документа, которая отражает его семантику.
Таким образом, для документов разных типов/предметных областей логическая структура отличается.
Например, художественные книги состоят из глав, научные статьи состоят из секций, подсекций и т.д., законы могут состоять из глав, статей, пунктов и т.п.
Тем не менее, в работе~\cite{dori1997representation} доказывается возможность построения такой логической структуры,
которая является общей для всех документов вне зависимости от их языка, предметной области, формата, внешнего вида, и т.д.

Выделение структуры общего вида может оказаться полезным по многим причинам.
Во-первых, она может служить основой для получения более специфической логической структуры, характерной для документов определённого типа.
Во-вторых, при анализе большого документа произвольного вида человеку или машине удобнее работать с его частями,
на которые желательно разбить документ осмысленным образом.
В-третьих, структурирование текстовых документов любого типа и формата позволяет хранить информацию в некотором общем представлении,
с помощью которого легче массово обрабатывать большое количество различных неструктурированных данных.

Ряд работ~\cite{summers1998automatic,liang1999document,mao2003document,doucet2011setting},
в том числе работа, предлагающая общую для всех документов логическую структуру~\cite{dori1997representation}, используют иерархическую структуру для представления документа.
Подобная структура является деревом, у корого листьями могут быть символы, слова, строки, параграфы и т.д., а сам документ -- это корень дерева.
Оглавление документа также является одним из частных случаев иерархической логической структуры документа, в которой листьями служат целые главы.
По этой причине задачу выделения логической структуры документа часто сводят к задаче выделения оглавления.

Нахождение структурированного вида документа не всегда детерминированно в силу того, что не существует формальных правил написания документов.
В таком случае возникает необходимость в использовании современных методов машинного обучения для того,
чтобы извлекать логическую структуру из документов при её недетерминированности.