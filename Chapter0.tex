\section{Введение}
\label{sec:Chapter0} \index{Chapter0}

Рукописные записи повсеместно используются в нашей повседневной жизни, как правило, в заметках, списках или других коротких текстах.
До изобретения печатного станка в XV веке, рукописи были единственным способом передачи и сохранения информации различного рода.
Поэтому огромное количество информации содержится в рукописном виде в исторических документах.
Кроме того, рукописный текст систематически используется в других областях, например, в написании конспектов на академических занятиях или на деловых встречах.
Несмотря на появление новых технологий, таких как компьютеры, планшеты и смартфоны, записи от руки по-прежнему являются предпочтительными для многих людей,
для того, чтобы зафиксировать свои идеи или мысли, по крайней мере, на начальном этапе.
Помимо быстроты и удобства использования, рукописный текст в настоящее время может применяться для заполнения различных бланков и заявлений,
что является важной составляющей в работе многих организаций.

Несмотря на повсеместность и удобство в использовании рукописных записей различного рода, этот способ не является предпочтительным в настоящее время.
Основной недостаток рукописей связан с исключительной трудностью их цифровизации с целью более удобного хранения, структуризации и распространения информации.
В современном мире большая часть процессов работы с данными автоматизирована, с ними работают компьютеры.
Однако компьютер не умеет работать с аналоговыми данными, такими как изображения рукописного текста, эти данные должны быть представлены в понятном для машины виде.
В этом контексте способность распознавать и оцифровывать содержимое рукописного текста необходима для извлечения из него необходимой информации.

В добавление к вышеперечисленным приложениям, проблема распознавания рукописного текста также возникает при работе
с фотографиями <<реального мира>> (real-world scenes), которые могут помимо прочего включать плакаты, этикетки или таблички с именами.
Эта потребность может также возникнуть при извлечении текста из видео, например, чтобы предоставить автономным системам вождения возможность понимать названия,
которые появляются в информационных знаках на дороге.
Задача распознавания текста на фотографиях из <<реального мира>> называется Scene text recognition (STR) и не имеет точного перевода на русский язык.
Эта задача наряду с обыкновенным распознаванием рукописного текста является активной областью исследований~\cite{zhu2016scene}.

Автоматическое распознавание рукописного текста (handwritten text recognition, HTR) -- задача, которая решается в течение уже довольно продолжительного времени~\cite{plamondon2000online}.
Она состоит в автоматическом переводе изображений, содержащих рукописный текст, в символьное представление.
Это относительно простая задача для людей, но очень сложная задача для моделирования компьютером.
В настоящее время она рассматривается в общем виде как еще нерешенная и активно исследуется.
Несколько конференций и периодических изданий специализируются непосредственно на задаче распознавания рукописного текста, например,
Международная конференция по ограничениям в распознавании рукописного ввода (ICFHR),
а другие выделяют задачу распознавания рукописей как одну из многочисленных решаемых подзадач,
например Международная конференция по анализу и распознаванию документов (ICDAR).
Существуют также специализированные журналы в этой области, такие как Международный журнал по анализу и распознаванию документов (IJDAR).
Кроме того, задача распознавания рукописного текста широко изучалась в литературе~\cite{plamondon2000online,sueiras2021continuous}.
Несмотря на недавние достижения, значительные различия в почерках разных людей и неточный характер написания символов
делают эту задачу сложной, поэтому она не решена до сих пор.

Отдельно следует сказать о распознавании рукописного текста на русском языке.
Большинство исследований в области распознавания ведется для текста на английском языке, либо на языке, основой которого являются латинские символы.
Вследствие этого, намного проще найти и наборы данных, и методы решения задачи для текста на таких языках.
При этом существует лишь несколько работ, посвящённых распознаванию текста на кириллице~\cite{abdallah2020attention,shonenkov2021stackmix},
равно как и небольшое количество эталонных наборов данных, используемых для сравнения результатов с другими методами.
Всё это добавляет дополнительные сложности на пути решения задачи к уже существующим многочисленным проблемам.

В настоящее время развивается целое направление науки под названием \textit{машинное обучение}.
Суть машинного обучения состоит в автоматической разработке компьютерных алгоритмов,
способных воспроизводить решение сложной задачи на основе предыдущего опыта, полученного из анализа данных~\cite{prakash2021pattern}.
В течение многих лет машинное обучение успешно применялось к структурированным данным в виде таблиц.
Каждая запись в таблице соответствовала обучающему примеру, а каждый столбец был числовым атрибутом, с помощью которого строился алгоритм в виде математической формулы.
Однако алгоритмы машинного обучения не могли учиться на неструктурированных данных, таких как изображения или текст.
Ситуация резко изменилась с появлением глубокого обучения.
Глубокое обучение использует нейронные модели, в которых функция обучения определяется как композиция
большого количества более простых функций, называемых слоями~\cite{goodfellow2016deep}.
Подобные модели глубокого обучения успешно используются в особо сложных задачах, таких как классификация изображений, распознавание речи или машинный перевод.

Для того чтобы успешно применять машинное и глубокое обучение при решении различного рода задач, необходимо иметь обучающие данные.
В случае глубокого обучения в силу сложности моделей и большого количества обучаемых параметров, данных должно быть очень много и они должны быть разнообразны.
В этом случае хорошо себя показывают методы аугментации данных, однако в рамках рассматриваемой задачи их может быть недостаточно,
так как рукописный текст сильно вариативен как по почерку, так и по количеству слов, которые могут быть написаны.
Одной из основных причин отсутствия хорошей универсальной модели распознавания рукописного текста является то, что размер обучающих данных недостаточно велик.
При этом изменение существующих данных в процессе обучения путем искажения изображений не спасает от переобучения на одинаковых словах и соединениях символов.
Таким образом, в рамках задачи распознавания рукописного текста также решается задача расширения обучающего набора данных.

Как правило, расширение обучающего набора данных осуществляется генерацией изображений рукописного текста с помощью рукописных шрифтов.
Эта методика характерна для англоязычных текстов, для которых создано большое количество разнообразных шрифтов.
Несмотря на относительную простоту реализации этого метода, в литературе не упоминается его использование в контексте русского языка.
Это может быть связано с меньшим разнообразием доступных шрифтов, а также слабой изученностью данной темы в принципе.
Помимо метода генерации данных с помощью шрифтов, применяют также и другие техники,
зачастую требующие обучения специализированных моделей машинного обучения~\cite{shonenkov2021stackmix,fogel2020scrabblegan}.
Изучение эффективности данных методов может позволить улучшить качество обучаемых моделей распознавания рукописного текста,
в частности, на русском языке.
