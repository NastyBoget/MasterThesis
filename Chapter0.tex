\section{Введение}
\label{sec:Chapter0} \index{Chapter0}


\todo[inline]{В этой части надо описать предметную область, задачу из которой вы будете решать, объяснить её актуальность (почему надо что-то делать сейчас?).
Здесь же стоит ввести определения понятий, которые вам понадобятся в постановке задачи.}


Рукописные записи повсеместно используются в нашей повседневной жизни как правило в заметках, списках или других коротких текстах.
До изобретения печатного станка в XV веке, рукописи были единственным способом передачи и сохранения информации различного рода.
Поэтому огромное количество информации содержится в рукописном виде в исторических документах.
Кроме того, рукописный текст систематически используется в других областях, таких как написание конспектов на академических занятиях или на деловых встречах.
Несмотря на появление новых технологий, таких как компьютеры, планшеты и смартфоны, записи от руки по-прежнему являются предпочтительными для многих людей, для того чтобы зафиксировать свои идеи или мысли, по крайней мере, на начальном этапе.
Помимо быстроты и удобства использования, рукописный текст в настоящее время может применяться для заполнения различных бланков и заявлений, что является важной составляющей в работе многих организаций.


