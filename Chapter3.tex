\section{Исследование и построение решения задачи}
\label{sec:Chapter3} \index{Chapter3}

\subsection{Описание иерархической структуры документа общего вида}
\label{subsec:structuredescription}

\subsection{Разработка метода построения иерархической структуры общего вида}
\label{subsec:extractmethod}

Согласно подсекции~\ref{subsec:structuredescription} дерево документа, которое необходимо построить,
является упорядоченным, а в своих узлах содержит строки текстового документа.
Для того, чтобы построить такое дерево для произвольного документа, необходимо:
\begin{enumerate}
    \item Получить упорядоченное множество строк документа;
    \item Для каждой строки найти строку-родителя, которая находится выше по иерархии, и строки-потомки, которые структурно вложены по отношению к данной строке.
\end{enumerate}
Для нахождения иерархии строк, то есть того, какая строка является предком, а какая потомком,
можно применять метод попарного сравнения строк.
Этот метод подразумевает одновременное последовательное прохождение по строкам документа и построение дерева документа.
При этом, из каждой новой рассматриваемой строки документа нужно сформировать новый узел дерева, а для этого определить место вставки узла.
При анализе документа мы опираемся на тот факт, что некоторые строки являются более значимыми,
то есть являются составными частями более значимых структурных элементов, чем другие строки.
Например, строка-заголовок документа является более значимой, чем простая текстовая строка.
Если мы умеем определять, какая из двух строк более значима, то мы сможем определить место вставки очередного узла в дерево документа.
Если строка менее значима, чем предыдущая, то нужно добавить новый лист, если же строка оказалась более значимой,
то её необходимо проходить вверх по иерархии дерева и искать строку-родителя для нахождения места вставки.
Более точно алгоритм описывается следующим образом:
\todo[inline]{Строго сформулировать алгоритм на псевдокоде}

Согласно указанному алгоритму, основная задача, которая должна быть решена -- это определение для пары строк документа, какая из них является более значимой.
Эта задача не является полностью формализованной, так как у документов разных типов оформление сильно различается.
Однако есть некоторые общие правила по выделению структурированных элементов в виде заголовков и списков.
Таким образом, для решения данной задачи можно применить методы машинного обучения.

Для применения машинного обучения необходимо составить набор данных, на которых можно обучаться.
Так как поставленная задача еще никем не решалась, создание набора данных становится одной из подзадач, которые надо решить.
При этом, надо принять во внимание тот факт, что документы в наборе данных должны различаться по структуре, оформлению и предметной области.
Процесс разметки не является очевидным, поэтому для создания набора размеченных данных требуется создать программную систему,
которая бы позволила размечать (сравнивать) пары строк так, как это делается в алгоритме построения дерева документа.
На основании созданного и размеченного набора документов далее можно обучить алгоритм машинного обучения,
который позволил бы сравнивать строки документа по значимости, а значит и находить место вставки каждой строки-узла в итоговое дерево документа.

Обобщая всё вышесказанное, для решения задачи построения дерева документа предлагается следующая последовательность действий:
\begin{enumerate}
    \item Сформировать набор документов с разным оформлением и предметной областью;
    \item Создать систему разметки, позволяющую динамически создавать задания для сравнения строк документа;
    \item Организовать получение упорядоченного списка строк из документов с дополнительной информацией, необходимой для системы разметки;
    \item Применить алгоритм машинного обучения на размеченных данных для сравнения строк документа;
    \item Основываясь на списке выделенных из документа строк, а также обученном алгоритме сравнения строк, построить дерево документа согласно описанному алгоритму \tofo[inline]{здесть ссылка на алгоритм}.
\end{enumerate}

\subsubsection{Описание набора данных}

\subsubsection{Система разметки}

\subsubsection{Процесс разметки документов}

\subsubsection{Обучение алгоритма сравнения строк}


\subsection{Разработка метода использования иерархической структуры общего вида}
\label{subsec:usemethod}

\subsection{Оценку качества реализованного метода}
\label{subsec:evaluation}