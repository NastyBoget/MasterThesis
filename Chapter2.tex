\section{Обзор существующих решений}
\label{sec:Chapter2} \index{Chapter2}

% Описание задачи и её типов
Распознавание рукописного текста (HTR) позволяет переводить рукописный текст в цифровой формат.
Основная сложность задачи заключается в значительной вариативности рукописного текста.
Почерк каждого человека отличается от других людей (межличностная изменчивость),
и даже один и тот же человек пишет одно и то же слово по-разному (внутриличностная изменчивость)~\cite{sueiras2021continuous}.
Такие факторы, как скорость письма, размер шрифта, тип используемой бумаги и ручки и даже эмоциональное состояние,
еще больше увеличивают эту изменчивость.

В зависимости от характера исходных данных существуют два основных направления исследования задачи.
С одной стороны, решается задача распознавания рукописного текста на бумажной странице,
отсканированной в цифровое изображение, называемая автономным (оффлайн) распознаванием рукописного текста.
С другой стороны, существует так называемое онлайн-распознавание рукописного ввода,
заключающееся в распознавании рукописного текста по непрерывным данным о положениях $(x, y)$ ручки,
полученных при написании этого текста.
Эти данные обычно получают путем ввода непосредственно на сенсорном экране.
Мы ограничимся исследованием задачи оффлайн распознавания рукописного текста.

% распознавание рукописного текста в контексте анализа документов в целом
Проблема распознавания рукописного текста в автономном режиме открывает несколько направлений исследований, которые еще далеко не полностью решены.
Для распознавания рукописного текста в документе необходимо, чтобы изначально были определены части изображения, содержащие рукописный текст.
Кроме того, при необходимости текст может быть сегментирован в строки и слова.
Изображение текста может встречаться в таблицах или формах, перекрываться с другими элементами страницы, такими как изображения, схемы или графики.
Текст может быть написан на разных языках с разными наборами символов (т.е. алфавитами), например на китайском, арабском или японском.

Нахождение различных типов элементов (например, рукописный текст, печатный текст, графики и т.д.)
в отсканированных изображениях документов называется ~\textit{анализом макета} (layout analysis).
Это отдельная сложная и до сих пор нерешённая задача, которая широко исследуется в настоящее время~\cite{binmakhashen2019document}.

% сегментация текста
После нахождения текста на изображении, его распознавание обычно включает в себя несколько шагов~\cite{plamondon2000online}.
В самом общем случае изображение с текстом может содержать параграф, состоящий из нескольких строк.
В этом случае чаще всего текст разбивается на строки.
Однако сегментация строк не является тривиальной из-за наклона строк, наклона символов внутри строки и того факта,
что некоторые символы в соседних строках могут перекрываться другими.
Модели распознавания могут быть применены непосредственно к текстовым строкам, либо их можно разбить на слова, и распознавание выполнять на уровне слов.
Большинство современных алгоритмов HTR можно применять либо к строкам, либо к словам, которые являются частным случаем коротких строк.

% наборы данных и аугментация
Для обучения моделей распознавания рукописного текста с использованием методов глубокого обучения
требуются наборы данных изображений рукописного текста, должным образом аннотированные текстом, присутствующего в каждом изображении.
Существует несколько общедоступных наборов данных, некоторые из них содержат изображения отдельных символов,
другие содержат изображения слов, строк и абзацев.
В любом случае объем аннотированных данных, доступных для обучения моделей, ограничен и требует рассмотрения возможности
использования стратегий увеличения данных (аугментации) для обучения моделей.

В следующих секциях описано, каким образом измеряется качество работы методов распознавания рукописного текста.
Кроме того, описаны наборы данных с рукописным текстом на русском языке, а также методы предобработки и аугментации данных.
В заключение, перечисляются основные типы нейросетевых моделей, входящие в состав различных архитектур нейронных сетей,
применяемых для решения задачи распознавания рукописного текста.
Модели в основном разрабатываются для текстов на английском языке, тем не менее существуют решения для русского языка,
которые также описаны в контексте других моделей.


\subsection{Метрики оценки качества}
\label{subsec:evaluation-metrics}

Двумя основными метриками, обычно используемыми для оценки моделей распознавания рукописного текста на уровне слов и строк,
являются \textit{частота ошибок символов} (Character Error Rate, CER) и \textit{частота ошибок слов} (Word Error Rate, WER).

CER измеряет расстояние Левенштейна~\cite{levenshtein1966binary} между предсказанной и реальной последовательностью символов слова.
Расстояние Левенштейна, также иногда называемое расстоянием редактирования,
представляет собой метрику для измерения разницы между двумя последовательностями.
Неформально расстояние Левенштейна между двумя словами (предсказание модели и реальное слово) --
это минимальное количество вставок, удалений или замен, необходимых для преобразования предсказания в правильное слово,
делённое на длину правильного слова, как показано в уравнении~(\ref{eq:cer}).

\begin{equation}
    \label{eq:cer}
    CER(prediction,real)=\frac{substitutions+insertions+deletes}{len(real)}
\end{equation}

Частота ошибок в словах (WER) определяется аналогично CER путем вычисления минимального количества вставок, замен и удалений слов,
необходимых для перехода от текстовой строки, предсказанной моделью, к реальной текстовой строке.
В случае, когда распознавание выполняется на уровне отдельных слов, WER представляет собой процент слов,
правильно предсказанных моделью, что соответствует точности модели, описанной в уравнении~(\ref{eq:wer}).
\begin{equation}
    \label{eq:wer}
    accuracy=\frac{1}{N}\sum_{i=j}c_{ij}
\end{equation}


\subsection{Наборы данных}
\label{subsec:datasets}

Существует не так много общедоступных наборов данных на русском языке для обучения моделей и сравнения результатов.
На текущий момент известно два набора данных со словами на кириллице:
\begin{itemize}
    \item Cyrillic Handwriting Dataset\footnote{\url{https://www.kaggle.com/datasets/constantinwerner/cyrillic-handwriting-dataset}};
    \item HKR\footnote{\url{https://github.com/abdoelsayed2016/HKR_Dataset}}.
\end{itemize}

Кроме того, найдено два набора данных русских символов, являющиеся не очень популярными:
\begin{itemize}
    \item CoMNIST\footnote{\url{https://github.com/GregVial/CoMNIST}};
    \item База сегментированных рукописных символов\footnote{\url{https://drive.google.com/folderview?id=0B0EQUc5HmgcGS0l2RDlKenlpNnc&usp=sharing}}.
\end{itemize}

\begin{table}[H]
    \centering
    \begin{tabular}{|p{4cm}|p{7cm}|p{3cm}|}
        \hline
        \textbf{Название} & \textbf{Описание} & \textbf{Размер} \\
        \hline
        \hline
        Cyrillic Handwriting Dataset & Набор русских текстов длиной $\leqslant$ 40 символов, собранный из различных интернет-ресурсов & train=72286 test=1544 \\
        \hline
        HKR & Набор из русских (95\%) и казахских (5\%) слов и предложений: ключевые слова, поэмы и алфавит & train=45470 val=9359 test1=5057 test2=5057 \\
        \hline
        CoMNIST & Символы -- русские заглавные буквы, собран при помощи краудсорсинга & более 28000 \\
        \hline
        База сегментированных рукописных символов & Обширная база из строчных и прописных рукописных символов, а также цифр & всего 120750 символов \\
        \hline
    \end{tabular}
    \caption{Наборы данных с кириллицей}
    \label{tab:datasets}
\end{table}


\subsection{Методы предобработки данных}
\label{subsec:preprocessing}

Несмотря на явное различие в написании текста на кириллице и латиннице, в рукописных текстах есть некоторое сходство,
которое позволяет применять похожие методы предобработки данных к текстам на разных языках.
В кириллических рукописных текстах аналогично латинским присутствует наклон символов,
текст пишется слева направо -- соответственно может встречаться наклон строк.
Кроме того, в строках могут встречаться символы, выходящие за пределы основной строки --
например, заглавные символы или символы с верхними или нижними петлями.
Текст, как правило, пишется на бумаге, которая может иметь дефекты, некоторые недостатки может иметь и сам рукописный текст.
Поэтому имеет смысл рассматривать методы предобработки данных, используемые для текстов на латиннице (в частности, для английского языка).
Далее будут более подробно описаны некоторые методики, позволяющие улучшить качество входных данных и привести их к более нормализованному виду.

\subsubsection{Удаление шума и бинаризация}
\label{subsubsec:binarization}

Оцифрованное изображение рукописного текста подвержено множеству источников шума, поэтому даже человеку иногда сложно его распознать.
Бумага может содержать следы, может быть не совсем белой или испорченной.
Если лист бумаги тонкий, может быть виден также текст, написанный обратной стороне (эффект просвечивания).
В процессе оцифровки на изображении могут появиться артефакты, вызванные, загрязнениями сканера.
Шумоподавление изображения -- это первый шаг в обработке и стандартизации изображения.
Его цель состоит в том, чтобы получить изображение в градациях серого, в котором текст имеет четкие черные штрихи,
а фон не содержит элементов, влияющих на предсказания модели.

Крайний случай устранения шумов на изображении -- бинаризация.
В результате бинаризации изображение становится чёрно-белым, без градаций серого.
Пример предполагаемых результатов работы бинаризации представлен на рисунке~\ref{fig:binarization}.
\begin{figure}[H]
    \centering
    \includegraphics[width=0.6\textwidth]{img/binarization}
    \caption{Пример документов до (слева) и после (справа) бинаризации}
    \label{fig:binarization}
\end{figure}

Существует большое количество работ, посвящённых решению задачи бинаризации документов, а также обзоров методов~\cite{mustafa2018binarization}.
В частности, один из самых простых и эффективных методов -- использование адаптивного порога бинаризации, вычисляющегося для небольших участков изображения.


\subsubsection{Исправление наклона строки и символов}
\label{subsubsec:slope-slant-correction}

\textit{Наклон строки} -- это наклон текстовой строки документа относительно горизонтальной линии.
Как правило, он появляется, когда текст пишется на пустой странице без предварительной разлиновки.
На верхнем изображении рисунка~\ref{fig:slope-slant} показан пример наклона строки с положительным углом.
Коррекция наклона может выполняться на уровне страницы, а также на уровне строки или даже на уровне слова.

\textit{Наклон символов} -- распространённое свойство, используемое в процессе обучения письму.
Именно поэтому он очень популярен и является одним из основных источников вариативности рукописного текста,
что затрудняет его распознавание с помощью автоматических систем.
Идентификация и исправление наклона являются критически важными аспектами распознавания рукописного текста,
поскольку многие алгоритмы распознавания, использующие изображения рукописного текста в качестве входных данных,
обычно используют подход, который анализирует изображение столбец за столбцом.
К таким моделям можно отнести полносвязные нейронные сети (многослойный перцептрон, рекуррентные нейронные сети),
у моделей, использующих свёрточные сети, такой проблемы нет~\cite{sueiras2021continuous}.

В любом случае коррекция наклона символов обеспечивает снижение вариативности рукописного текста,
что облегчает его распознавание вне зависимости от типа используемой модели.
Поэтому исправление наклона символов, наряду с шумоподавлением, является самым распространенным методом предварительной обработки при решении задачи распознавания рукописного текста.
Коррекция наклона символов обычно выполняется в два этапа: определение угла наклона линий, а затем применение преобразования для коррекции этого угла.
Зачастую, исправление наклона символов иногда частично выполняется одновременно с исправлением наклона строк,
поскольку исправление угла наклона строки изменяет угол наклона символов на ту же величину.
Пример исправления наклона строки с последующим исправлением наклона символов показан на рисунке~\ref{fig:slope-slant}.

\begin{figure}[H]
    \centering
    \includegraphics[width=0.5\textwidth]{img/slant-slope}
    \caption{Пример исправления наклона строки и символов (внизу исправленный вариант)}
    \label{fig:slope-slant}
\end{figure}

Одним из самых популярных методов коррекции наклона строки является преобразование Хафа~\cite{duda1972use} из-за его надёжности и простоты.
Однако этот метод является вычислительно сложным.
Поэтому несколько авторов предложили варианты, которые уменьшают размер пространства Хафа~\cite{pal1996improved,boukharouba2017new}, хотя вычислительные затраты остаются высокими.

В качестве альтернативы другие авторы определяют угол наклона строки, оценивая линию, которая лучше всего соответствует набору пикселей на изображении.
Например, в работе~\cite{gupta2014efficient} используется линейная регрессия координат x, y пикселей в основной области текста.

Другой широко используемый метод, как для определения наклона строки, так и для определения наклона символов, основан на профилях проекций, например~\cite{kavallieratou2002skew} и~\cite{pastor2004projection}.
Горизонтальная проекция текстовых пикселей используется для определения угла наклона строки, а вертикальная проекция используется для определения угла наклона символов.
Подобные методы довольно чувствительны к присутствию шума на изображении, и для их применения необходимо сначала использовать методы упомянутые в разделе~\ref{subsubsec:binarization}.

Помимо метода, использующего профили проекций, существуют и другие методы исправления угла наклона символов.
В работе~\cite{gupta2012novel} предложен алгоритм определения угла наклона, который основан на способности фильтра Габора обнаруживать направленные текстуры.
В той же работе описан второй метод оценки угла наклона, который использует преобразование Фурье для преобразования изображения слова в спектр Фурье.
Повторение точек вдоль заданного направления активирует частотное пространство в его перпендикулярном направлении, и это направление соответствует углу наклона.
Другой пример можно в работе~\cite{vinciarelli2001new}, где метод определения угла наклона символов основан на гипотезе о том,
что изображение слова наклоняется, когда количество столбцов, содержащих непрерывные штрихи, максимально.


\subsection{Аугментация данных}
\label{subsec:augmentation}


\subsection{Типы нейросетевых моделей}
\label{subsec:networks-description}


\subsection{Архитектуры нейронных сетей, используемые для решения задачи}
\label{subsec:networks-architectures}

