\section{Заключение}
\label{sec:Chapter5} \index{Chapter5}

В результате проделанной работы были найдены и исследованы новейшие методики по распознаванию рукописного текста на изображениях.
В контексте распознавания рукописного текста на русском языке были изучены и описаны имеющиеся общедоступные эталонные наборы данных
для обучения моделей распознавания и оценки их качества.
Кроме того, был выявлен набор техник, применяющихся для расширения имеющихся обучающих наборов данных,
в силу того, что имеющихся данных недостаточно для получения удовлетворительного результата распознавания.

По результатам изучения способов генерации дополнительных обучающих данных выбрано три способа создания синтетических изображений рукописного текста:
\begin{itemize}
    \item с помощью рукописных шрифтов;
    \item с помощью склейки новых слов из нарезок от имеющихся слов (Stackmix);
    \item с помощью генеративно-состязательной сети.
\end{itemize}

Методы генерации с помощью генеративно-состязательной сети и на основе алгоритма Stackmix были использованы для генерации
дополнительных наборов данных в рамках сравнительного анализа данных методов.
В дополнение, был реализован и использован метод генерации изображений рукописного текста на русском языке на основе шрифтов,
так как подобные методы не упоминаются в научной литературе в контексте русского языка.
Расширяющие наборы, полученные тремя способами на основе двух эталонных наборов данных, доступны для загрузки
и использования\footnote{Наборы данных synthetic\_hkr, stackmix\_hkr, gan\_hkr, synthetic\_cyrillic, stackmix\_cyrillic, gan\_cyrillic на сайте \url{https://huggingface.co}}.
Кроме того, доступна реализация генератора рукописного текста, а также набор рукописных шрифтов\footnote{\url{https://github.com/NastyBoget/HandwritingGeneration}}.

По результатам анализа существующих архитектур моделей распознавания были отобраны модели, показывающие одни из лучших
результаты в рамках исследуемой области: модель разметки последовательности с механизмом внимания и модель с архитектурой трансформер.
Проведено сравнение эффективности методов расширения обучающих наборов данных путем обучения выбранных моделей
на наборах данных, дополненных сгенерированными изображениями, и сравнения качества распознавания обученных моделей на тестовых наборах.
Реализация проведенных экспериментов находится в открытом доступе\footnote{\url{https://github.com/NastyBoget/hrtr}} и доступна для воспроизведения.

Полученные результаты позволяют сделать вывод о том, что реализованный метод генерации синтетических данных для дополнения
эталонных обучающих наборов сравним по эффективности с существующими предложенными ранее методами --
в среднем он дает прирост на 10\% в точности распознавания текста.
Тем не менее, предложенный в данной работе метод не требует таких значительных вычислительных ресурсов и обучения дополнительных моделей,
следовательно имеет преимущество перед другими подходами.