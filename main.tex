\documentclass[oneside,senior,etd]{BYUPhysForDegree}

\usepackage[utf8]{inputenc}
\usepackage{rotating} 

\usepackage[russian]{babel}
\usepackage{amsfonts} % Пакеты для математических символов и теорем
\usepackage{amstext}
\usepackage{amssymb}
\usepackage{amsthm}
\usepackage{graphicx} % Пакеты для вставки графики
\usepackage{subfig}
\usepackage[unicode]{hyperref}
\usepackage[nottoc]{tocbibind} % Для того, чтобы список литературы отображался в оглавлении
\usepackage{verbatim} % Для вставок заранее подготовленного текста в режиме as-is
\usepackage{listings}

\usepackage{commath}
\newcommand\Tau{\mathcal{T}}
\newcommand{\R}{\mathbb{R}}
\usepackage{color}
\usepackage[colorinlistoftodos, prependcaption]{todonotes}
\usepackage{multirow}
\usepackage[T1]{fontenc}
\newcommand*{\MyIndent}{\hspace*{0.2cm}}%

\Chair{Кафедра системного программирования}
\Lab{~}
\Year{2023}
  \Month{Май}
  \City{Москва}
  \AuthorText{Автор}
  \Author{Богатенкова Анастасия Олеговна}
  \AuthorEng{}
  \AcadGroup{628}
  \TitleTop{Распознавание рукописного текста на русском языке}
  %\TitleMiddle{}
  \TitleBottom{} % leave empty if you don't need it
  \TitleTopEng{Handwritten Russian text recognition}
  \TitleBottomEng{} % leave empty if you don't need it  
  %\docname{Курсовая работа}
  %\docname{Выпускная квалификационная работа}
  \docname{Магистерская диссертация}
  \Advisor{Корухова Людмила Сергеевна}
  \AdvisorDegree{к.ф.-м.н.}
  \Consultant{Беляева Оксана Владимировна}
  \ConsultantDegree{}
  
\Abstract{Задача автоматического распознавания рукописного текста является важной составляющей в процессе анализа электронных документов,
  однако её решение все еще далеко от идеала.
  Одной из основных причин сложности распознавания рукописного текста на русском языке является недостаточное количество данных, используемых для обучения моделей распознавания.
  При этом, для русского языка проблема встаёт более остро и усугубляется большим разнообразием сложных почерков.
  В данной работе исследуется влияние различных методов расширения обучающего набора данных на качество моделей распознавания.}

\AbstractEng{Automatic handwriting recognition is an important component in the process of electronic documents analysis, but its solution is still far from ideal.
One of the main reasons for the complexity of russian handwriting recognition is the insufficient amount of data used to train recognition models.
Moreover, for the Russian language the problem is more acute and is exacerbated by a large variety of complex handwriting.
In this work, it's studied the influence of training dataset expanding methods on the quality of recognition models.}

%%%% DON'T change this. It is here because .sty does not support cyrillic cp properly %%%%
\University{Московский государственный университет имени М.В. Ломоносова}
\Faculty{Факультет вычислительной математики и кибернетики}
\GrText{гр.}
\AdvisorText{Научный руководитель}
\ConsultantText{Научный консультант}
\AbstractText{Аннотация}

\begin{document}
\fixmargins
\makepreliminarypages

\oneandhalfspace

\tableofcontents

\section{Введение}
\label{sec:Chapter0} \index{Chapter0}


\todo[inline]{В этой части надо описать предметную область, задачу из которой вы будете решать, объяснить её актуальность (почему надо что-то делать сейчас?).
Здесь же стоит ввести определения понятий, которые вам понадобятся в постановке задачи.}


Рукописные записи повсеместно используются в нашей повседневной жизни как правило в заметках, списках или других коротких текстах.
До изобретения печатного станка в XV веке, рукописи были единственным способом передачи и сохранения информации различного рода.
Поэтому огромное количество информации содержится в рукописном виде в исторических документах.
Кроме того, рукописный текст систематически используется в других областях, таких как написание конспектов на академических занятиях или на деловых встречах.
Несмотря на появление новых технологий, таких как компьютеры, планшеты и смартфоны, записи от руки по-прежнему являются предпочтительными для многих людей, для того чтобы зафиксировать свои идеи или мысли, по крайней мере, на начальном этапе.
Помимо быстроты и удобства использования, рукописный текст в настоящее время может применяться для заполнения различных бланков и заявлений, что является важной составляющей в работе многих организаций.


 % Введение
\section{Постановка задачи}
\label{sec:Chapter1} \index{Chapter1}

Задача распознавания рукописного текста (HTR) заключается в разработке моделей и алгоритмов,
способных преобразовывать изображение рукописного текста в цифровой формат (последовательность символов).

Математическая постановка задачи выглядит следующим образом.

\begin{itemize}
    \item[] Пусть $x^{m \times n \times c}$ -- входное изображение шириной $m$ и высотой $n$, $c$ -- число каналов ($c=1$ для черно-белого изображения).
    \item[] Пусть $y^t=(y_1,\dots,y_t),~y_i \in A,~i=1,\dots t$ -- выходная последовательность символов из алфавита $A$.
    \item[] Пусть $X=\{x^{m \times n \times c},~m,n>0,~c\in\{1,3\}\},~Y=\{y^t,~0<t\leqslant T\}$ -- множества входных изображений и выходных последовательностей соответственно.
\end{itemize}

Задача распознавания рукописного текста состоит в определении отображения:
\[ X \rightarrow Y \]
задающего для каждого изображения рукописного текста его цифровое представление в виде последовательности символов.

Согласно главе~\ref{sec:Chapter0} в решении данной задачи хорошо себя зарекомендовали модели глубокого обучения.
Такие модели требуют обучения, соответственно необходим большой по объему и вариативный обучающий набор данных.
Как правило, для расширения набора данных используют аугментацию данных -- случайное преобразование имеющихся данных,
а также генерацию синтетических данных, обладающих достаточной реалистичностью.
При этом, именно создание исскусственных данных может существенно влиять на результат обучения в рамках текущей задачи в силу того,
что оно позволяет генерировать новые сочетания символов/слов и предотвращает переобучение модели на конкретных наборах слов.
Поэтому исследование эффективности методов генерации синтетических данных может помочь улучшить результаты обучаемых моделей.

Как было сказано в главе~\ref{sec:Chapter0}, одним из наиболее распространенных методов расширения обучающего набора данных
является генерация изображений текста на основе рукописных шрифтов.
Тем не менее, в научной литературе не упоминается применение данного метода для распознавания рукописного текста на русском языке.
Соответственно, нет информации и об его реализации.

Таким образом, в силу вышеописанных рассуждений, предлагается выполнить следующее:
\begin{itemize}
    \item Исследовать существующие обучающие наборы данных -- изображений рукописных слов на русском языке, методы их предобработки и аугментации;
    \item Исследовать архитектуры нейросетевых моделей распознавания рукописного текста, выбрать лучшие для дополнительных исследований;
    \item Разработать и реализовать метод генерации дополнительного обучающего набора данных на основе рукописных шрифтов;
    \item Исследовать влияние различных методов генерации дополнительного набора данных на качество предсказания моделей;
    \item Разработать и реализовать метод, позволяющий предсказывать по изображению рукописного слова текст, представленный на изображении;
    \item Провести оценку качества реализованного метода.
\end{itemize}
 % Постановка задачи
\section{Обзор существующих решений}
\label{sec:Chapter2} \index{Chapter2}

% Описание задачи и её типов
Распознавание рукописного текста (HTR) позволяет переводить рукописный текст в цифровой формат.
Основная сложность задачи заключается в значительной вариативности рукописного текста.
Почерк каждого человека отличается от других людей (межличностная изменчивость),
и даже один и тот же человек пишет одно и то же слово по-разному (внутриличностная изменчивость)~\cite{sueiras2021continuous}.
Такие факторы, как скорость письма, размер шрифта, тип используемой бумаги и ручки и даже эмоциональное состояние,
еще больше увеличивают эту изменчивость.

В зависимости от характера исходных данных существуют два основных направления исследования задачи.
С одной стороны, решается задача распознавания рукописного текста на бумажной странице,
отсканированной в цифровое изображение, называемая автономным (оффлайн) распознаванием рукописного текста.
С другой стороны, существует так называемое онлайн-распознавание рукописного ввода,
заключающееся в распознавании рукописного текста по непрерывным данным о положениях $(x, y)$ ручки,
полученных при написании этого текста.
Эти данные обычно получают путем ввода непосредственно на сенсорном экране.
Мы ограничимся исследованием задачи оффлайн распознавания рукописного текста.

% распознавание рукописного текста в контексте анализа документов в целом
Проблема распознавания рукописного текста в автономном режиме открывает несколько направлений исследований, которые еще далеко не полностью решены.
Для распознавания рукописного текста в документе необходимо, чтобы изначально были определены части изображения, содержащие рукописный текст.
Кроме того, при необходимости текст может быть сегментирован в строки и слова.
Изображение текста может встречаться в таблицах или формах, перекрываться с другими элементами страницы, такими как изображения, схемы или графики.
Текст может быть написан на разных языках с разными наборами символов (т.е. алфавитами), например на китайском, арабском или японском.

Нахождение различных типов элементов (например, рукописный текст, печатный текст, графики и т.д.)
в отсканированных изображениях документов называется ~\textit{анализом макета} (layout analysis).
Это отдельная сложная и до сих пор нерешённая задача, которая широко исследуется в настоящее время~\cite{binmakhashen2019document}.

% сегментация текста
После нахождения текста на изображении, его распознавание обычно включает в себя несколько шагов~\cite{plamondon2000online}.
В самом общем случае изображение с текстом может содержать параграф, состоящий из нескольких строк.
В этом случае чаще всего текст разбивается на строки.
Однако сегментация строк не является тривиальной из-за наклона строк, наклона символов внутри строки и того факта,
что некоторые символы в соседних строках могут перекрываться другими.
Модели распознавания могут быть применены непосредственно к текстовым строкам, либо их можно разбить на слова, и распознавание выполнять на уровне слов.
Большинство современных алгоритмов HTR можно применять либо к строкам, либо к словам, которые являются частным случаем коротких строк.

% наборы данных и аугментация
Для обучения моделей распознавания рукописного текста с использованием методов глубокого обучения
требуются наборы данных изображений рукописного текста, должным образом аннотированные текстом, присутствующего в каждом изображении.
Существует несколько общедоступных наборов данных, некоторые из них содержат изображения отдельных символов,
другие содержат изображения слов, строк и абзацев.
В любом случае объем аннотированных данных, доступных для обучения моделей, ограничен и требует рассмотрения возможности
использования стратегий увеличения данных (аугментации) для обучения моделей.

В следующих секциях описано, каким образом измеряется качество работы методов распознавания рукописного текста.
Кроме того, описаны наборы данных с рукописным текстом на русском языке, а также методы предобработки и аугментации данных.
В заключение, перечисляются основные типы нейросетевых моделей, входящие в состав различных архитектур нейронных сетей,
применяемых для решения задачи распознавания рукописного текста.
Модели в основном разрабатываются для текстов на английском языке, тем не менее существуют решения для русского языка,
которые также описаны в контексте других моделей.


\subsection{Метрики оценки качества}
\label{subsec:evaluation-metrics}

Двумя основными метриками, обычно используемыми для оценки моделей распознавания рукописного текста на уровне слов и строк,
являются \textit{частота ошибок символов} (Character Error Rate, CER) и \textit{частота ошибок слов} (Word Error Rate, WER).

CER измеряет расстояние Левенштейна~\cite{levenshtein1966binary} между предсказанной и реальной последовательностью символов слова.
Расстояние Левенштейна, также иногда называемое расстоянием редактирования,
представляет собой метрику для измерения разницы между двумя последовательностями.
Неформально расстояние Левенштейна между двумя словами (предсказание модели и реальное слово) --
это минимальное количество вставок, удалений или замен, необходимых для преобразования предсказания в правильное слово,
делённое на длину правильного слова, как показано в уравнении~(\ref{eq:cer}).

\begin{equation}
    \label{eq:cer}
    CER(prediction,real)=\frac{substitutions+insertions+deletes}{len(real)}
\end{equation}

Частота ошибок в словах (WER) определяется аналогично CER путем вычисления минимального количества вставок, замен и удалений слов,
необходимых для перехода от текстовой строки, предсказанной моделью, к реальной текстовой строке.
В случае, когда распознавание выполняется на уровне отдельных слов, WER представляет собой процент слов,
правильно предсказанных моделью, что соответствует точности модели, описанной в уравнении~(\ref{eq:wer}).
\begin{equation}
    \label{eq:wer}
    accuracy=\frac{1}{N}\sum_{i=j}c_{ij}
\end{equation}


\subsection{Наборы данных}
\label{subsec:datasets}

Существует не так много общедоступных наборов данных на русском языке для обучения моделей и сравнения результатов.
На текущий момент известно два набора данных со словами на кириллице:
\begin{itemize}
    \item Cyrillic Handwriting Dataset\footnote{\url{https://www.kaggle.com/datasets/constantinwerner/cyrillic-handwriting-dataset}};
    \item HKR\footnote{\url{https://github.com/abdoelsayed2016/HKR_Dataset}}.
\end{itemize}

Кроме того, найдено два набора данных русских символов, являющиеся не очень популярными:
\begin{itemize}
    \item CoMNIST\footnote{\url{https://github.com/GregVial/CoMNIST}};
    \item База сегментированных рукописных символов\footnote{\url{https://drive.google.com/folderview?id=0B0EQUc5HmgcGS0l2RDlKenlpNnc&usp=sharing}}.
\end{itemize}

\begin{table}[H]
    \centering
    \begin{tabular}{|p{4cm}|p{7cm}|p{3cm}|}
        \hline
        \textbf{Название} & \textbf{Описание} & \textbf{Размер} \\
        \hline
        \hline
        Cyrillic Handwriting Dataset & Набор русских текстов длиной $\leqslant$ 40 символов, собранный из различных интернет-ресурсов & train=72286 test=1544 \\
        \hline
        HKR & Набор из русских (95\%) и казахских (5\%) слов и предложений: ключевые слова, поэмы и алфавит & train=45470 val=9359 test1=5057 test2=5057 \\
        \hline
        CoMNIST & Символы -- русские заглавные буквы, собран при помощи краудсорсинга & более 28000 \\
        \hline
        База сегментированных рукописных символов & Обширная база из строчных и прописных рукописных символов, а также цифр & всего 120750 символов \\
        \hline
    \end{tabular}
    \caption{Наборы данных с кириллицей}
    \label{tab:datasets}
\end{table}


\subsection{Методы предобработки данных}
\label{subsec:preprocessing}

Несмотря на явное различие в написании текста на кириллице и латиннице, в рукописных текстах есть некоторое сходство,
которое позволяет применять похожие методы предобработки данных к текстам на разных языках.
В кириллических рукописных текстах аналогично латинским присутствует наклон символов,
текст пишется слева направо -- соответственно может встречаться наклон строк.
Кроме того, в строках могут встречаться символы, выходящие за пределы основной строки --
например, заглавные символы или символы с верхними или нижними петлями.
Текст, как правило, пишется на бумаге, которая может иметь дефекты, некоторые недостатки может иметь и сам рукописный текст.
Поэтому имеет смысл рассматривать методы предобработки данных, используемые для текстов на латиннице (в частности, для английского языка).
Далее будут более подробно описаны некоторые методики, позволяющие улучшить качество входных данных и привести их к более нормализованному виду.

\subsubsection{Удаление шума и бинаризация}
\label{subsubsec:binarization}

Оцифрованное изображение рукописного текста подвержено множеству источников шума, поэтому даже человеку иногда сложно его распознать.
Бумага может содержать следы, может быть не совсем белой или испорченной.
Если лист бумаги тонкий, может быть виден также текст, написанный обратной стороне (эффект просвечивания).
В процессе оцифровки на изображении могут появиться артефакты, вызванные, загрязнениями сканера.
Шумоподавление изображения -- это первый шаг в обработке и стандартизации изображения.
Его цель состоит в том, чтобы получить изображение в градациях серого, в котором текст имеет четкие черные штрихи,
а фон не содержит элементов, влияющих на предсказания модели.

Крайний случай устранения шумов на изображении -- бинаризация.
В результате бинаризации изображение становится чёрно-белым, без градаций серого.
Пример предполагаемых результатов работы бинаризации представлен на рисунке~\ref{fig:binarization}.
\begin{figure}[H]
    \centering
    \includegraphics[width=0.6\textwidth]{img/binarization}
    \caption{Пример документов до (слева) и после (справа) бинаризации}
    \label{fig:binarization}
\end{figure}

Существует большое количество работ, посвящённых решению задачи бинаризации документов, а также обзоров методов~\cite{mustafa2018binarization}.
В частности, один из самых простых и эффективных методов -- использование адаптивного порога бинаризации, вычисляющегося для небольших участков изображения.


\subsubsection{Исправление наклона строки и символов}
\label{subsubsec:slope-slant-correction}

\textit{Наклон строки} -- это наклон текстовой строки документа относительно горизонтальной линии.
Как правило, он появляется, когда текст пишется на пустой странице без предварительной разлиновки.
На верхнем изображении рисунка~\ref{fig:slope-slant} показан пример наклона строки с положительным углом.
Коррекция наклона может выполняться на уровне страницы, а также на уровне строки или даже на уровне слова.

\textit{Наклон символов} -- распространённое свойство, используемое в процессе обучения письму.
Именно поэтому он очень популярен и является одним из основных источников вариативности рукописного текста,
что затрудняет его распознавание с помощью автоматических систем.
Идентификация и исправление наклона являются критически важными аспектами распознавания рукописного текста,
поскольку многие алгоритмы распознавания, использующие изображения рукописного текста в качестве входных данных,
обычно используют подход, который анализирует изображение столбец за столбцом.
К таким моделям можно отнести полносвязные нейронные сети (многослойный перцептрон, рекуррентные нейронные сети),
у моделей, использующих свёрточные сети, такой проблемы нет~\cite{sueiras2021continuous}.

В любом случае коррекция наклона символов обеспечивает снижение вариативности рукописного текста,
что облегчает его распознавание вне зависимости от типа используемой модели.
Поэтому исправление наклона символов, наряду с шумоподавлением, является самым распространенным методом предварительной обработки при решении задачи распознавания рукописного текста.
Коррекция наклона символов обычно выполняется в два этапа: определение угла наклона линий, а затем применение преобразования для коррекции этого угла.
Зачастую, исправление наклона символов иногда частично выполняется одновременно с исправлением наклона строк,
поскольку исправление угла наклона строки изменяет угол наклона символов на ту же величину.
Пример исправления наклона строки с последующим исправлением наклона символов показан на рисунке~\ref{fig:slope-slant}.

\begin{figure}[H]
    \centering
    \includegraphics[width=0.5\textwidth]{img/slant-slope}
    \caption{Пример исправления наклона строки и символов (внизу исправленный вариант)}
    \label{fig:slope-slant}
\end{figure}

Одним из самых популярных методов коррекции наклона строки является преобразование Хафа~\cite{duda1972use} из-за его надёжности и простоты.
Однако этот метод является вычислительно сложным.
Поэтому несколько авторов предложили варианты, которые уменьшают размер пространства Хафа~\cite{pal1996improved,boukharouba2017new}, хотя вычислительные затраты остаются высокими.

В качестве альтернативы другие авторы определяют угол наклона строки, оценивая линию, которая лучше всего соответствует набору пикселей на изображении.
Например, в работе~\cite{gupta2014efficient} используется линейная регрессия координат x, y пикселей в основной области текста.

Другой широко используемый метод, как для определения наклона строки, так и для определения наклона символов, основан на профилях проекций, например~\cite{kavallieratou2002skew} и~\cite{pastor2004projection}.
Горизонтальная проекция текстовых пикселей используется для определения угла наклона строки, а вертикальная проекция используется для определения угла наклона символов.
Подобные методы довольно чувствительны к присутствию шума на изображении, и для их применения необходимо сначала использовать методы упомянутые в разделе~\ref{subsubsec:binarization}.

Помимо метода, использующего профили проекций, существуют и другие методы исправления угла наклона символов.
В работе~\cite{gupta2012novel} предложен алгоритм определения угла наклона, который основан на способности фильтра Габора обнаруживать направленные текстуры.
В той же работе описан второй метод оценки угла наклона, который использует преобразование Фурье для преобразования изображения слова в спектр Фурье.
Повторение точек вдоль заданного направления активирует частотное пространство в его перпендикулярном направлении, и это направление соответствует углу наклона.
Другой пример можно в работе~\cite{vinciarelli2001new}, где метод определения угла наклона символов основан на гипотезе о том,
что изображение слова наклоняется, когда количество столбцов, содержащих непрерывные штрихи, максимально.


\subsection{Аугментация данных}
\label{subsec:augmentation}


\subsection{Типы нейросетевых моделей}
\label{subsec:networks-description}


\subsection{Архитектуры нейронных сетей, используемые для решения задачи}
\label{subsec:networks-architectures}

 % Обзор существующих решений
\section{Исследование и построение решения задачи}
\label{sec:Chapter3} \index{Chapter3}

\subsection{Описание иерархической структуры документа общего вида}
\label{subsec:structuredescription}

\subsection{Разработка метода построения иерархической структуры общего вида}
\label{subsec:extractmethod}

Согласно подсекции~\ref{subsec:structuredescription} дерево документа, которое необходимо построить,
является упорядоченным, а в своих узлах содержит строки текстового документа.
Для того, чтобы построить такое дерево для произвольного документа, необходимо:
\begin{enumerate}
    \item Получить упорядоченное множество строк документа;
    \item Для каждой строки найти строку-родителя, которая находится выше по иерархии, и строки-потомки, которые структурно вложены по отношению к данной строке.
\end{enumerate}
Для нахождения иерархии строк, то есть того, какая строка является предком, а какая потомком,
можно применять метод попарного сравнения строк.
Этот метод подразумевает одновременное последовательное прохождение по строкам документа и построение дерева документа.
При этом, из каждой новой рассматриваемой строки документа нужно сформировать новый узел дерева, а для этого определить место вставки узла.
При анализе документа мы опираемся на тот факт, что некоторые строки являются более значимыми,
то есть являются составными частями более значимых структурных элементов, чем другие строки.
Например, строка-заголовок документа является более значимой, чем простая текстовая строка.
Если мы умеем определять, какая из двух строк более значима, то мы сможем определить место вставки очередного узла в дерево документа.
Если строка менее значима, чем предыдущая, то нужно добавить новый лист, если же строка оказалась более значимой,
то её необходимо проходить вверх по иерархии дерева и искать строку-родителя для нахождения места вставки.
Более точно процесс построения дерева документа описывается алгоритмом~\ref{alg:treebuilding}.

\begin{algorithm}
    \hspace*{\algorithmicindent} \textbf{Input:} $L$ -- список из $N$ строк документа \\
    \hspace*{\algorithmicindent} \textbf{Output:} $T$ -- дерево документа
    \begin{algorithmic}
        \State $current\_line \gets L[0]$
        \State $T \gets make\_node(current\_line)$ \Comment{Создание из строки документа узла дерева}
        \State $current\_node \gets T$
        \State $i \gets 1$
        \While{$i < N$}
            \State $next\_line \gets L[i]$
            \State $cmp \gets compare\_lines(current\_line, next\_line)$ \Comment{Сравнение строк}
            \If{$cmp = less$} \Comment{$next\_line < current\_line$}
                \State $new\_node \gets make\_node(next\_line)$
                \State $add\_child(current\_node, new\_node)$
            \ElsIf{$cmp = greater$} \Comment{$next\_line > current\_line$}
                \If{$current\_node$ является корнем}
                    \State $new\_node \gets make\_node(next\_line)$
                    \State $T \gets new\_node$
                    \State $add\_child(new\_node, current\_node)$
                \Else
                    \State $current\_node \gets get\_parent(current\_node)$
                    \State $current\_line \gets get\_line(current\_node)$
                    \State \textbf{continue}
                \EndIf
            \Else \Comment{$next\_line = current\_line$}
                \State $parent\_node \gets get\_parent(current\_node)$
                \State $new\_node \gets make\_node(next\_line)$
                \State $add\_child(parent\_node, new\_node)$
            \EndIf
            \State $current\_line \gets next\_line$
            \State $current\_node \gets new\_node$
            \State $i \gets i + 1$
        \EndWhile
    \end{algorithmic}
    \caption{Алгоритм построения дерева документа}
    \label{alg:treebuilding}
\end{algorithm}

Согласно указанному алгоритму, основная задача, которая должна быть решена -- это определение для пары строк документа, какая из них является более значимой.
Эта задача не является полностью формализованной, так как у документов разных типов оформление сильно различается.
Однако есть некоторые общие правила по выделению структурированных элементов в виде заголовков и списков.
Таким образом, для решения данной задачи можно применить методы машинного обучения.

Для применения машинного обучения необходимо составить набор данных, на которых можно обучаться.
Так как поставленная задача еще никем не решалась, создание набора данных становится одной из подзадач, которые надо решить.
При этом, надо принять во внимание тот факт, что документы в наборе данных должны различаться по структуре, оформлению и предметной области.
Процесс разметки не является очевидным, поэтому для создания набора размеченных данных требуется создать программную систему,
которая бы позволила размечать (сравнивать) пары строк так, как это делается в алгоритме построения дерева документа.
На основании созданного и размеченного набора документов далее можно обучить алгоритм машинного обучения,
который позволил бы сравнивать строки документа по значимости, а значит и находить место вставки каждой строки-узла в итоговое дерево документа.

Обобщая всё вышесказанное, для решения задачи построения дерева документа предлагается следующая последовательность действий:
\begin{enumerate}
    \item Сформировать набор документов с разным оформлением и предметной областью;
    \item Создать систему разметки, позволяющую динамически создавать задания для сравнения строк документа;
    \item Организовать получение упорядоченного списка строк из документов с дополнительной информацией, необходимой для системы разметки;
    \item Применить алгоритм машинного обучения на размеченных данных для сравнения строк документа;
    \item Основываясь на списке выделенных из документа строк, а также обученном алгоритме сравнения строк, построить дерево документа согласно описанному алгоритму~\ref{alg:treebuilding}.
\end{enumerate}

\subsubsection{Описание набора данных}

В набор данных включены несколько типов документов, написанных на разных языках:
\begin{itemize}
    \item Нормативно-правовые акты на русском языке;
    \item Нормативно-правовые акты на армянском языке;
    \item Научные статьи на английском языке;
    \item Финансовые документы на английском и французском языках.
\end{itemize}

Основная информация о документах представлена в таблице~\ref{tab:dataset}.
\begin{table}
    \begin{center}
        \begin{tabular}{|p{4cm}|p{3cm}|p{5cm}|p{2.5cm}|}
            \hline
            \textbf{Предметная область} & \textbf{Язык} & \textbf{Источник} & \textbf{Количество документов} \\
            \hline
            \hline
            Нормативно-правовые акты & русский & Официальный портал правовой информации\tablefootnote{\url{http://publication.pravo.gov.ru/}} & 20 \\
            \hline
            Нормативно-правовые акты & армянский & Информационная система\tablefootnote{\url{https://www.arlis.am/}} и электронное правительство\tablefootnote{\url{https://www.e-gov.am/en/}} Республики Армения & 20 \\
            \hline
            Научные статьи & английский & Архив научных статей с открытым доступом\tablefootnote{\url{https://arxiv.org/}} & 20 \\
            \hline
            Финансовые документы & английский, французский & Сайт соревнований FinTOC-2021\tablefootnote{\url{http://wp.lancs.ac.uk/cfie/fintoc2021/}} & 10 \\
            \hline
        \end{tabular}
    \end{center}
    \caption{Описание набора данных}
    \label{tab:dataset}
\end{table}

Все документы представлены в формате PDF как с текстовым слоем, так и без него.
Среди них присутствуют документы разного оформления: одноколоночные и двухколоночные, написанные шрифтом разного размера, в некоторых документах присутствуют нетекстовые документы такие как печати, таблицы, изображения.
Примеры страниц документов разных типов представлены на рисунке~\ref{fig:documentexamples}.

\begin{figure}[h]
    \centering
    \begin{minipage}[h]{0.22\linewidth}
    \center{\frame{\includegraphics[height=0.22\textheight]{img/russian}} \\ а) НПА на русском}
    \end{minipage}
    \begin{minipage}[h]{0.22\linewidth}
    \center{\frame{\includegraphics[height=0.22\textheight]{img/armenian}} \\ б) НПА на армянском}
    \end{minipage}
    \begin{minipage}[h]{0.22\linewidth}
    \center{\frame{\includegraphics[height=0.22\textheight]{img/article}} \\ в) Научная статья}
    \end{minipage}
    \begin{minipage}[h]{0.22\linewidth}
    \center{\frame{\includegraphics[height=0.22\textheight]{img/fintoc}} \\ г) Финансовый документ}
    \end{minipage}
    \caption{Примеры страниц документов из набора данных}
    \label{fig:documentexamples}
\end{figure}

\todo[inline]{Описать поподробнее и мб добавить статистики}

\subsubsection{Система разметки}

Поставленная задача подразумевает сравнение двух строк между собой.
Для разметчика эту задачу можно поставить как задачу классификации изображений:
\begin{enumerate}
    \item Система выдаёт разметчику изображение, состоящее из двух страниц документа, на каждой из страниц выделена одна из строк;
    \item Разметчик определяет, какая из выделенных строк <<важнее>>, классифицируя таким образом изображение.
\end{enumerate}

Для того, чтобы процесс разметки походил на процесс выполнения алгоритма построения дерева документа, необходимо уметь решать динамически, какую пару строк сравнивать следующей.
А для этого в системе разметки должен быть прописан алгоритм, похожий на описанный выше алгоритм построения дерева (см. алгоритм~\ref{alg:treebuilding}).
Кроме того, нужно уметь каким-то образом выделять строки на страницах документа.
Для документов в формате PDF можно сохранять страницы как изображения, а для строк находить координаты рамок, ограничивающих текст строки (bounding boxes).
Таким образом, для системы разметки документ достаточно представить как список прочитанных строк со следующей информацией:
\begin{itemize}
    \item устойчивый уникальный идентификатор строки;
    \item путь до изображения страницы с данной строкой;
    \item координаты ограничивающей рамки для строки.
\end{itemize}
При наличии такой информации можно создавать изображение, состоящее из двух страниц документа, на каждой их которых одна строка обведена в рамку.
Уникальный идентификатор строки нужен для навигации по списку строк при выборе очередной пары строк, которые нужно сравнить.

Обобщая всё перечисленное, система разметки должна предоставлять следующие возможности:
\begin{itemize}
    \item Формирование входных данных для разметки, то есть получение из документа списка строк с
    идентификатором, путём до изображения страницы и координатами ограничивающей рамки, а также сохранение страниц документа как отдельных изображений.
    \item Формирование изображений, которые нужно классифицировать.
    Изображения формируются обведением на изображениях страниц нужных строк в рамки, а также соединением две страницы в одно изображение.
    \item Формирование заданий для разметчика, т. е. последовательный проход по полученным из документа строкам и в соответствии с алгоритмом построения дерева и результатами разметки предыдущих строк формирование следующих заданий для разметки.
    Задания для разметки получаются с помощью формирования изображений.
    \item Пользователь (разметчик) должен иметь возможность взаимодействовать с системой разметки и сохранять результаты своего труда.
\end{itemize}

Программная реализация такой системы описана в главе~\ref{sec:Chapter4}.
Предлагается ввести 4 класса, на которые нужно классифицировать изображения: вторая строка <<меньше>> (less), <<больше>> (greater) или <<равна>> (equal) первой, либо одна из строк не является текстовой строкой, т.е. <<другое>> (other).
Эти классы, а также правила, по которым разметчик должен снабжать метками изображения, описаны в подсекции~\ref{subsubsec:labelingprocess}.
Здесь же схематично приведён алгоритм~\ref{alg:labeling}, по которому в зависимости от предыдущих результатов разметки формируются новые задания.

\begin{algorithm}[h]
    \hspace*{\algorithmicindent} \textbf{Input:} $L$ -- список строк документа с дополнительной информацией, \\
    \hspace*{\algorithmicindent} $T$ -- список размеченных заданий из $id_1,~id_2,~label$, \\
    \hspace*{\algorithmicindent} где $id_1,~id_2$ -- индексы сравниваемых строк в списке $L$, $label$ -- метка \\
    \hspace*{\algorithmicindent} \textbf{Output:} $new\_task$ -- новое задание для разметки
    \begin{algorithmic}
        \State $last\_label \gets get\_last\_label(T)$ \Comment{Метка для последнего задания}
        \State $last\_id_1, last\_id_2 \gets get\_last\_ids(T)$ \Comment{Номера строк в списке $L$ из задания}
        \If{$last\_label = other$}
            \State $line \gets find\_line(T)$ \Comment{Поиск строки в заданиях с меткой $\neq other$}
            \State $new\_task \gets make\_task(line, last\_id_2 + 1)$ \Comment{Формирование задания}
        \ElsIf{$last\_label = greater$}
            \State $line \gets find\_line(last\_id_1, T)$ \Comment{Поиск строки > строки с номером $last\_id_1$}
            \State $new\_task \gets make\_task(line, last\_id_2)$ \Comment{Формирование задания}
        \Else \Comment{$last\_label = less$ или $last\_label = equal$}
            \State $new\_task \gets make\_task(last\_id_2, last\_id_2 + 1)$ \Comment{Формирование задания}
    \end{algorithmic}
    \caption{Алгоритм формирования задания для разметки}
    \label{alg:labeling}
\end{algorithm}



\subsubsection{Процесс разметки документов}
\label{subsubsec:labelingprocess}

Для того, чтобы процесс разметки всего набора данных был осмысленным и согласованным с поставленной задачей, а результаты отностительно однозначными, необходимо провести некоторую предварительную работу:
\begin{enumerate}
    \item Сформулировать правила правильной разметки (манифест);
    \item Дать нескольким разметчикам небольшое количество заданий для разметки и получить результаты;
    \item Сравнить результаты и в случае грубых несоответствий вернуться к первому шагу и начать всё сначала, пересмотрев манифест.
\end{enumerate}
Если разметчики достигли определённого уровня согласованности при выполнении задания, значит правила разметки адекватны и можно приступать к разметке всего набора данных.

Таким образом, для осуществления процесса разметки нужно составить манифест и выбрать метрику для сравнения результатов.

\paragraph{Правила разметки.} Строго сформулированные правила разметки вкратце описываются следующим образом:
\begin{itemize}
    \item Задание состоят из пар изображений страниц одного документа, на каждой из страниц одна из строк выделена.
    Нужно определить, больше, меньше или равна вторая строка документа первой строке и присвоить одну из меток: <<greater>>, <<less>>, <<equal>>, <<other>>.
    \item Строки документа могут иметь разное выделение, для которого могут использовать отступ, выравнивание; полужирный, подчёркнутый или курсивный шрифт; текст из заглавных букв, префиксы для элементов списка в начале строки.
    \item Строки считаются равными (equal) по уровню вложенности, если они являются составляющими одинаково (похожим образом) выделенных структурных элементов. Например, заголовки двух глав, два элемента одного списка, две строки одного текстового блока.
    \item Вторая строка меньше первой (less), если первая строка – заголовок (сильно выделена), а вторая – список или текст; либо первая строка – список, а вторая текст; либо первая строка визуально более выделена, чем вторая (жирностью, отступом и т. д.); либо это многоуровневый список.
    Для greater всё аналогично с точностью до наоборот.
    \item Тип <<other>> выбирается, когда хотя бы одна из строк является сноской (номером страницы, и т.д.), либо рукописным текстом, либо вообще не текстом (картинка, текстовый блок из нескольких строк и т. д.)
\end{itemize}

\paragraph{Метрика для сравнения результатов разметки.} Для сравнения древовидных структур документов (например, содержания) в нескольких соревнованиях~\cite{doucet2013icdar,fintoc2019,fintoc2020,el2021financial} использовалась функция следующего вида:
\[ precision = \frac{correct}{correct + added + mismatch} \]
\[ recall = \frac{correct}{correct + missed + mismatch} \]
\[ F1 = \frac{2 \cdot precision \cdot recall}{precision + recall}, \]
где
\begin{itemize}
    \item $correct$ --
    \item $mismatch$ --
    \item $added$ --
    \item $missed$ --
\end{itemize}

\subsubsection{Обучение алгоритма сравнения строк}


\subsection{Разработка метода использования иерархической структуры общего вида}
\label{subsec:usemethod}


\subsection{Оценку качества реализованного метода}
\label{subsec:evaluation} % Исследование и построение решения задачи
\section{Описание практической части}
\label{sec:Chapter4} \index{Chapter4}

Так как тема данной работы связана с активным использованием техник глубокого обучения, реализация всех методов и экспериментов
сделана на языке программирования \textit{Python}, который предоставляет все необходимые библиотеки, а именно:
\begin{itemize}
    \item \textit{Pillow}, \textit{opencv-python} -- для работы с изображениями, в т.ч. отрисовки шрифтов на изображении;
    \item \textit{albumentations}, \textit{augmixations} -- содержат реализацию основных методов аугментации изображений;
    \item \textit{pandas} -- для удобной работы с табличными данными, в частности для работы с информацией о размеченных наборах данных;
    \item \textit{scikit-learn}, \textit{scikit-image}, \textit{torch}, \textit{torchvision}, \textit{transformers} -- библиотеки с реализацией алгоритмов машинного обучения, а также основных метрик качества;
    \item \textit{datasets} -- для удобной загрузки наборов данных с \textit{huggingface.co}.
\end{itemize}

В рамках выполнения поставленной задачи было создано два публичных репозитория на платформе \textit{github.com}:
\begin{itemize}
    \item \textit{HandwritingGeneration}\footnote{\url{https://github.com/NastyBoget/HandwritingGeneration}}, содержащий код метода генерации синтетических изображений рукописного текста на основе шрифтов;
    \item \textit{hrtr}\footnote{\url{https://github.com/NastyBoget/hrtr)}}, содержащий код загрузки и обработки наборов данных, а также запуска необходимых экспериментов.
\end{itemize}

Помимо этого, для хранения и удобной загрузки созданных синтетических наборов данных использовалась платформа \textit{huggingface.co},
предоставляющая возможность бесплатно загружать и хранить файлы больших размеров.

В следующих секциях дано описание деталей реализации генерации изображений текста на основе рукописных шрифтов,
создания и публикации сгенерированных синтетических наборов данных, а также запуска экспериментов.


\subsection{Реализация генератора изображений рукописного текста}
\label{subsec:handwriting-generation}

В секции~\ref{subsec:synthetic} описан метод полу-автоматической генерации рукописных шрифтов, а также создания синтетических изображений на основе шрифтов.
Для этого был реализован набор модулей с классами-генераторами, доступный по ссылке \url{https://github.com/NastyBoget/HandwritingGeneration}.
На рисунке~\ref{fig:diagram_handwriting_generation} представлена диаграмма реализованных классов.

\begin{figure}[h!]
    \centering
    \includegraphics[width=\textwidth]{img/diagram_handwriting_generation}
    \caption{Диаграмма классов модуля генерации изображений рукописного текста}
    \label{fig:diagram_handwriting_generation}
\end{figure}

Показанные на рисунке~\ref{fig:diagram_handwriting_generation} классы позволяют осуществлять следующее:
\begin{itemize}
    \item \textit{TemplateDrawer} -- заполнение шаблона для приложения \textit{calligraphr}\footnote{\url{https://www.calligraphr.com}} для создания рукописных шрифтов;
    \item \textit{HandwritingGenerator} -- создание изображений рукописного текста с возможностью рандомизации стиля написания и фона;
    \item \textit{TextGenerator} -- генерация случайных текстов методом скачивания статей из Википедии\footnote{\url{https://ru.wikipedia.org}} и их очистки от неподдерживаемых символов;
    \item \textit{Generator} -- непосредственно генерация синтетического набора данных, включая генерацию текстов с их последующей отрисовкой.
\end{itemize}

Полученная реализация позволяет отрисовывать набор символов русского алфавита вместе с цифрами в среднем за 0.6 секунды
на компьютере MacBook Pro с процессором 1.4GHz Quad-Core Intel Core i5 и памятью 8GB 2133MHz LPDDR3.
Временная оценка работы генератора текста варьируется в зависимости от скорости работы сети, а также от длины статей, получаемых случайным образом.


\subsection{Реализация экспериментальной части}
\label{subsec:hrtr}

Реализация экспериментов, результаты которых описаны в секции~\ref{subsubsec:experiments_results}, находится в репозитории \url{https://github.com/NastyBoget/hrtr}.
В него включен отредактированный код моделей AttentionHTR и трансформера, а также их обучения и оценки качества распознавания.
Кроме того, важной частью является предварительная работа с данными -- их загрузка и приведение к унифицированному виду.
Для этого в репозитории есть специальный модуль, описанный в секции~\ref{subsubsec:datasets-processing}.
Далее более подробно описана реализация экспериментальной части.

\subsubsection{Создание и публикация дополнительных обучающих наборов данных}
\label{subsubsec:datasets-processing}

Прежде чем приступить к обучению моделей распознавания, необходимо иметь обучающий набор данных, представленный с специальном виде.
Для этого на языке \textit{Python} реализован модуль \textit{process\_datasets}, в котором каждый набор данных обрабатывается отдельным классом.
Диаграмма реализованных классов представлена на рисунке~\ref{fig:diagram_process_datasets}.

\begin{figure}[h!]
    \centering
    \includegraphics[width=\textwidth]{img/diagram_process_datasets}
    \caption{Диаграмма классов модуля загрузки и обработки наборов данных}
    \label{fig:diagram_process_datasets}
\end{figure}

Классы \textit{HKRDatasetProcessor} и \textit{CyrillicDatasetProcessor} используются для загрузки и обработки наборов данных HKR и Cyrillic Handwriting Dataset соответственно.
Класс \textit{SyntheticDatasetProcessor}, изображенный на рисунке~\ref{fig:diagram_process_datasets}, отвечает за обработку дополнительных наборов данных, сгенерированных автоматически.
В нём происходит загрузка наборов данных с \url{https://huggingface.co}, процесс генерации которых описан в главе~\ref{sec:Chapter3}, а именно:
\begin{itemize}
    \item Дополнение набора HKR сгенерированными данными с помощью шрифтов\footnote{\url{https://huggingface.co/datasets/nastyboget/synthetic_hkr}};
    \item Дополнение набора HKR сгенерированными данными с помощью Stackmix\footnote{\url{https://huggingface.co/datasets/nastyboget/stackmix_hkr}};
    \item Дополнение набора HKR сгенерированными данными с помощью ScrabbleGAN\footnote{\url{https://huggingface.co/datasets/nastyboget/gan_hkr}};
    \item Дополнение набора Cyrillic Handwriting Dataset сгенерированными данными с помощью шрифтов\footnote{\url{https://huggingface.co/datasets/nastyboget/synthetic_cyrillic}};
    \item Дополнение набора Cyrillic Handwriting Dataset сгенерированными данными с помощью Stackmix\footnote{\url{https://huggingface.co/datasets/nastyboget/stackmix_cyrillic}};
    \item Дополнение набора Cyrillic Handwriting Dataset сгенерированными данными с помощью ScrabbleGAN\footnote{\url{https://huggingface.co/datasets/nastyboget/gan_cyrillic}};
\end{itemize}

Для удобной загрузки и распаковки синтетических данных \textit{huggingface.co} предлагает API, которое позволяет с помощью
скрипта на языке \textit{Python} задать принцип, по которому формируется загружаемый набор данных.
На основе этого скрипта библиотека \textit{datasets} позволяет загружать и приводить данные к заданному виду при помощи одной строки кода.
Таким образом, для каждого из наборов был написан соответствующий скрипт, а класс \textit{SyntheticDatasetProcessor}
завершает более специфическую обработку данных в контексте других наборов.


\subsubsection{Обучение и оценка качества моделей распознавания}
\label{subsubsec:models-train-evaluation}

При наличии загруженных наборов данных можно приступать к обучению и оценке качества моделей распознавания рукописного текста.
Описание реализации архитектур двух моделей распознавания рукописного текста, их обучения и оценки качества представлено на рисунке~\ref{fig:diagram_hrtr}.

\begin{figure}[h!]
    \centering
    \includegraphics[width=\textwidth]{img/diagram_hrtr}
    \caption{Диаграмма пакетов для реализации обучения моделей и оценки качества}
    \label{fig:diagram_hrtr}
\end{figure}

Рисунок~\ref{fig:diagram_hrtr} описывает следующую структуру исходного кода:
\begin{itemize}
    \item Пакет \textit{utils} содержит вспомогательный код для логгирования процесса создания наборов данных и обучения,
    а также реализацию метрик оценки качества.
    Данный код написан самостоятельно для облегчения проведения всех необходимых экспериментов.
    \item Пакет \textit{datasets} содержит переопределение класса \textit{Dataset} библиотеки \textit{torch},
    сделанное для осуществления равномерной выборки данных их разных наборов в процессе обучения.
    Данный пакет содержит адаптеры, позволяющие специфичным образом обрабатывать данные для выбранных моделей распознавания.
    Помимо этого, в пакете находится фиксированный набор аугментаций, использующийся при обучении моделей,
    а также вспомогательная функция определения символьного набора данных.
    Данный пакет также реализован самостоятельно.
    \item Пакет \textit{attention\_model} содержит модифицированный код модели AttentionHTR\footnote{\url{https://github.com/dmitrijsk/AttentionHTR}}
    распознавания текста на английском языке.
    Составляющие пакета описывают архитектуру модели: модуль трансформации, модуль извлечения признаков resnet,
    модуль разметки последовательности bilstm, а также модуль декодирования attention.
    Модуль model позволяет собрать все предыдущие воедино, label\_converter используется для преобразования символов в численное представление и обратно,
    resize\_normalization используется для предобработки изображений перед их подачей на вход сети.
    Модули train и test используются для обучения модели и оценки результатов ее обучения соответственно.
    \item Пакет \textit{transformer\_model} содержит модифицированный код модели трансформер\footnote{\url{https://github.com/t0efL/end2end-HKR-research}}
    распознавания текста на английском и русском языках.
    Составляющие пакета convnext, transformer и model содержат реализацию архитектуры модели, data позволяет кодировать и декодировать символьные метки для изображений,
    custom\_functions, criterions и utils используются в процессе обучения модели, который реализован в модуле train.
    Там же находится код для оценки качества обученной модели.
\end{itemize}

Таким образом, реализованы все необходимые инструменты, использующиеся для проведения экспериментов и получения численных результатов
для сравнения методов генерации дополнительных обучающих наборов данных.

\subsection{Выводы}
\label{subsec:prac_conclusions}

В результате проделанной работы был реализован генератор изображений рукописного текста для русского языка на основе шрифтов\footnote{\url{https://github.com/NastyBoget/HandwritingGeneration}}.
Этот способ наряду с существующими реализованными методами был использован для получения дополнительных обучающих данных,
сформированные синтетические наборы находятся в открытом доступе на сайте \url{https://huggingface.co}.
Для проведения экспериментов реализованы модули обработки и унификации наборов данных, а также скрипты обучения и
оценки качества моделей распознавания рукописного текста.
Исходный код всех экспериментов находится в публичном доступе\footnote{\url{https://github.com/NastyBoget/hrtr}},
что позволяет воспроизвести полученные в работе результаты. % Описание Экспериментальной части
\section{Заключение}
\label{sec:Chapter5} \index{Chapter5}

В результате проделанной работы были найдены и исследованы новейшие методики по распознаванию рукописного текста на изображениях.
В контексте распознавания рукописного текста на русском языке были изучены и описаны имеющиеся общедоступные эталонные наборы данных
для обучения моделей распознавания и оценки их качества.
Кроме того, был выявлен набор техник, применяющихся для расширения обучающих наборов данных,
в силу того, что имеющихся данных недостаточно для получения удовлетворительного результата распознавания.

По результатам изучения способов генерации дополнительных обучающих данных для сравнения выбрано три способа создания синтетических изображений рукописного текста:
\begin{itemize}
    \item с помощью рукописных шрифтов;
    \item с помощью склейки новых слов из нарезок от имеющихся слов (Stackmix);
    \item с помощью генеративно-состязательной сети.
\end{itemize}

Методы генерации с помощью генеративно-состязательной сети и на основе алгоритма Stackmix были использованы для генерации
дополнительных наборов данных в рамках сравнительного анализа данных методов.
В дополнение, был реализован и использован метод генерации изображений рукописного текста на русском языке на основе шрифтов,
так как подобные методы не упоминаются в научной литературе в контексте русского языка.
Расширяющие наборы, полученные тремя способами на основе двух эталонных наборов данных, доступны для загрузки
и использования\footnote{Наборы данных synthetic\_hkr, stackmix\_hkr, gan\_hkr, synthetic\_cyrillic, stackmix\_cyrillic, gan\_cyrillic на сайте \url{https://huggingface.co}}.
Кроме того, доступна реализация генератора рукописного текста, а также набор рукописных шрифтов\footnote{\url{https://github.com/NastyBoget/HandwritingGeneration}}.

По результатам анализа существующих архитектур моделей распознавания были отобраны модели, показывающие одни из лучших
результаты в рамках исследуемой области: модель разметки последовательности с механизмом внимания и модель с архитектурой трансформер.
Проведено сравнение эффективности методов расширения обучающих наборов данных путем обучения выбранных моделей
на наборах данных, дополненных сгенерированными изображениями, и сравнения качества распознавания обученных моделей на тестовых наборах.
Реализация проведенных экспериментов находится в открытом доступе\footnote{\url{https://github.com/NastyBoget/hrtr}} и доступна для воспроизведения.

Полученные результаты позволяют сделать вывод о том, что реализованный метод генерации синтетических данных для дополнения
эталонных обучающих наборов сравним по эффективности с существующими предложенными ранее методами --
в среднем он дает прирост на 6\% в точности распознавания текста.
Тем не менее, предложенный в данной работе метод не требует таких значительных вычислительных ресурсов и обучения дополнительных моделей,
следовательно, имеет преимущество перед другими подходами. % Заключение

\nocite{*}
\bibliographystyle{gost71u} % Для соответствия требованиям об оформлении списка литературы
\bibliography{references}


\end{document}
