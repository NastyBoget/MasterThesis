\section{Постановка задачи}
\label{sec:Chapter1} \index{Chapter1}

Задача распознавания рукописного текста (HTR) заключается в разработке моделей и алгоритмов,
способных преобразовывать изображение рукописного текста в цифровой формат (последовательность символов).

Математическая постановка задачи выглядит следующим образом.

\begin{itemize}
    \item[] Пусть $x^{m \times n \times c}$ -- входное изображение шириной $m$ и высотой $n$, $c$ -- число каналов ($c=1$ для черно-белого изображения).
    \item[] Пусть $y^t=(y_1,\dots,y_t),~y_i \in A,~i=1,\dots t$ -- выходная последовательность символов из алфавита $A$.
    \item[] Пусть $X=\{x^{m \times n \times c},~m,n>0,~c\in\{1,3\}\},~Y=\{y^t,~0<t\leqslant T\}$ -- множества входных изображений и выходных последовательностей соответственно.
\end{itemize}

Задача распознавания рукописного текста состоит в определении отображения:
\[ X \rightarrow Y \]
задающего для каждого изображения рукописного текста его цифровое представление в виде последовательности символов.

Согласно главе~\ref{sec:Chapter0} в решении данной задачи хорошо себя зарекомендовали модели глубокого обучения.
Такие модели требуют обучения, соответственно необходим большой по объему и вариативный обучающий набор данных.
Как правило, для расширения набора данных используют аугментацию данных -- случайное преобразование имеющихся данных,
а также генерацию синтетических данных, обладающих достаточной реалистичностью.
При этом, именно создание исскусственных данных может существенно влиять на результат обучения в рамках текущей задачи в силу того,
что оно позволяет генерировать новые сочетания символов/слов и предотвращает переобучение модели на конкретных наборах слов.
Поэтому исследование эффективности методов генерации синтетических данных может помочь улучшить результаты обучаемых моделей.

Как было сказано в главе~\ref{sec:Chapter0}, одним из наиболее распространенных методов расширения обучающего набора данных
является генерация изображений текста на основе рукописных шрифтов.
Тем не менее, в научной литературе не упоминается применение данного метода для распознавания рукописного текста на русском языке.
Соответственно, нет информации и об его реализации.

Таким образом, в силу вышеописанных рассуждений, предлагается выполнить следующее:
\begin{itemize}
    \item Исследовать существующие обучающие наборы данных -- изображений рукописных слов на русском языке, методы их предобработки и аугментации;
    \item Исследовать архитектуры нейросетевых моделей распознавания рукописного текста, выбрать лучшие для дополнительных исследований;
    \item Разработать и реализовать метод генерации дополнительного обучающего набора данных на основе рукописных шрифтов;
    \item Исследовать влияние различных методов генерации дополнительного набора данных на качество предсказания моделей;
    \item Разработать и реализовать метод, позволяющий предсказывать по изображению рукописного слова текст, представленный на изображении;
    \item Провести оценку качества реализованного метода.
\end{itemize}
