\section{Постановка задачи}
\label{sec:Chapter1} \index{Chapter1}

\todo[inline]{Надо переделывать и дополнять}
Нахождение иерархической логической структуры общего вида требует спецификации извлекаемого представления.
Поэтому необходимо описать то, каким образом любой документ может быть представлен в виде дерева.
Задача состоит в формализации понятия дерева документа и реализации метода для автоматического его нахождения в любом документе.
Кроме того, нужно проверить полезность выделения структуры общего вида при анализе документов определённой предметной области.
В рамках поставленной задачи необходимо выполнить следующее:
\begin{itemize}
    \item Формально описать извлекаемую иерархическую структуру;
    \item Реализовать метод построения иерархической структуры общего вида для любых документов;
    \item Описать и реализовать метод использования иерархической структуры общего вида при
    получении специфической логической структуры, характерной для документов определённого типа.
    \item Провести оценку качества реализованного метода;
    \item Встроить реализованный метод в открытый проект по обработке документов\footnote{\url{https://github.com/ispras/dedoc}}.
\end{itemize}