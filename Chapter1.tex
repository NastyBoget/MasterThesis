\section{Постановка задачи}
\label{sec:Chapter1} \index{Chapter1}

Задача распознавания рукописного текста (HTR) заключается в разработке моделей и алгоритмов,
способных преобразовывать изображение рукописного текста в цифровой формат (последовательность символов).

Математическая постановка задачи выглядит следующим образом.

\begin{itemize}
    \item[] Пусть $x^{m \times n \times c}$ -- входное изображение шириной $m$ и высотой $n$, $c$ -- число каналов ($c=1$ для черно-белого изображения).
    \item[] Пусть $y^t=(y_1,\dots,y_t),~y_i \in A,~i=1,\dots t$ -- выходная последовательность символов из алфавита $A$.
    \item[] Пусть $X=\{x^{m \times n \times c},~m,n>0,~c\in\{1,3\}\},~Y=\{y^t,~0<t\leqslant T\}$ -- множества входных изображений и выходных последовательностей соответственно.

\end{itemize}

Задача распознавания рукописного текста состоит в определении отображения:
\[ X \rightarrow Y \]
задающего для каждого изображения рукописного текста его цифровое представление в виде последовательности символов.

Согласно главе~\ref{sec:Chapter0} в решении данной задачи хорошо себя зарекомендовали модели глубокого обучения,
ориентированные на декодирование последовательности.
Такие модели требуют обучения, соответственно необходим обучающий набор данных.

Таким образом, в рамках поставленной задачи необходимо выполнить следующее:
\begin{itemize}
    \item Сформировать и описать обучающий набор данных -- изображений рукописных слов на русском языке;
    \item Исследовать и сравнить различные архитектуры нейросетевых моделей распознавания рукописного текста;
    \item Исследовать влияние предобработки данных на качество предсказания моделей;
    \item Разработать и реализовать метод, позволяющий предсказывать по изображению рукописного слова текст, представленный на изображении;
    \item Провести оценку качества реализованного метода.
\end{itemize}